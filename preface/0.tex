首先恭喜大家通过了前面的考核,选择了加入视觉/算法组。   

机器视觉是当前计算机发展的一个重要的研究方向,它旨在实现机器人对环境进行更加详细地感知。
由于可见光在成像方面清晰度、成像速度方面等具有很好的优势,因此机器视觉的研究也逐渐成为计算机视觉领域的研究热点。
在技术发展过程中,计算机视觉的技术可以大体分为两个技术方向,即图像处理与机器学习。

\textbf{图像处理}是指通过计算机对图像进行分析、处理、识别、理解等,从而实现对图像的高效、准确的描述、分析和理解。
主要的方法是通过计算机算法对图像进行处理,如图像增强、图像分割、图像检索、图像配准、图像修复、图像检索、图像分类、图像检索、图像压缩等。
在实际的应用场景中具有性能开销小,运算速度快,开发成本低等优点。

\textbf{机器学习}是指通过神经网络对图像进行分析。今年来随着深度学习和卷积神经网络的发展,机器学习技术也越来越多的应用于计算机视觉中。
机器学习的主要方法是通过训练数据对计算机模型进行训练,使计算机能够对未知数据进行预测、分类、聚类、回归等。
在图像处理中,机器学习可以应用于图像分类、目标检测、图像分割、图像检索、图像配准、图像修复等。
也涌现出了一大批优秀的物体识别算法,如YOLO等。

在实际的应用场景中,两种方法不是互相孤立的,而是需要我们相互结合,取长补短,才能达到更好的效果。除此之外,
光有计算机视觉也是远远不够的,
在现实的生产生活中,制造机器人主要的目的是为了达到自动化,面对真正的项目,
一般都是在一个miniPC中实现控制理论和多个数据融合处理,之后配合若干下位机进行自动化地完成任务。
所以代码、算法以及一些控制理论只是我们的工具,真正的目标是多感官的融合和自动化的控制。

俗话说“工欲善其事,必先利其器”,只有我们对基本的编程语言,对计算机的行为逻辑,对硬件作用与联系有着深刻的理解,
才能更好地理解每一种算法的特点,更好地洞悉每种操作背后的原理。从而应用到实际,切实地提高我们的创新能力和解决问题的能力。

在我们的培训中,我们将由浅入深,细致地为大家讲解机器视觉的知识,让大家能够更加深入地理解图像处理与机器学习的原理,
并运用到实际的项目中,提升自己的能力。在接下来的学习中,我们将更加注重理论知识的学习,重点讲解机器视觉的理论基础,
以及如何运用到实际的项目中。

送同学们一首我非常喜欢的诗:

\textbf{天高云淡, 望断南飞雁。 不到长城非好汉, 屈指行程二万。} 

\textbf{六盘山上高峰, 红旗漫卷西风。 今日长缨在手, 何时缚住苍龙?}

最后,祝大家学习愉快,工作顺利!
