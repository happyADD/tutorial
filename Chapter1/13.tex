如果你真的碰巧忘记备份镜像了,而你修改的环境变量又真的无法修复回去,或者有修复它的时间还不如重装一遍,那还是老老实实卸载重装吧。当然,希望各位自己永远不要用到这里的方法,在每次配环境前一定养成备份镜像的好习惯。

还是再提醒一下,重装之前备份好原ubuntu系统上的工作数据。

进入windows系统,右键“此电脑”选择“管理”,点击“磁盘管理”进入分区界面。确认好ubuntu系统存放的分区并将其删除,注意一定要将所有ubuntu分区都删除掉,不要错选到windows的分区!

搜索CMD(或搜索命令行),并用管理员身份打开。输入
\begin{tcode}
	diskpart
	list disk
\end{tcode}

确定windows系统的EFI分区,一般来说第一个磁盘0就是。输入
\begin{tcode}
	select disk 0
	list partition
\end{tcode}

然后确认ubuntu系统所在的分区,一般是在类型栏第一个显示为“系统”的分区,给该分区分配可见盘符,输入
\begin{tcode}
	select partition 2
	assign letter=F
\end{tcode}

这时候可以在电脑上看到一个磁盘F,但是没有权限访问。

再搜索记事本,用管理员权限打开,点击“文件”->“打开”->“磁盘F”,可以在打开界面看见F盘中的EFI文件夹,进入里面删掉ubuntu这个文件夹(该文件夹中内容无法访问但是可以直接删除)。删除后关闭记事本,在CMD中关闭刚刚分配的F盘,输入
\begin{tcode}
	remove letter=F
\end{tcode}

提示成功删除后ubuntu就已经完整的删除了,接下来就按照一开始安装ubuntu系统的步骤再重新安装一遍就可以了。

经实测发现,这里的方法会删除电脑中所有的ubuntu系统,如果你的电脑中有两个ubuntu系统那么还请慎用。

